\documentclass[t]{beamer}
\usetheme{Boadilla}

\usepackage{animate}
\usepackage[english]{babel}
\usepackage[font=tiny]{caption}
\usepackage[outputdir=/home/wlt/.cache/gummi/]{minted}
\usepackage{multicol}
\usepackage{multirow}
\usepackage{setspace}
\usepackage{tikz}

\newcommand{\eg}{\textit{e}.\textit{g}., }
\newcommand{\ie}{\textit{i}.\textit{e}., }
\newcommand{\heading}[1]{{\it{\usebeamercolor[fg]{frametitle} #1}}}

\definecolor{osprey_blue}{cmyk}{1, .76, .1, .65}
\definecolor{osprey_gray}{cmyk}{.36, .29, .28, 0}
\definecolor{osprey_green}{cmyk}{.59, .02, 1, 0}

\setbeamercolor*{palette primary}{bg=osprey_blue, fg = white}
\setbeamercolor*{palette secondary}{bg=osprey_gray, fg = white}
\setbeamercolor*{palette tertiary}{bg=osprey_blue, fg = white}
\setbeamercolor{structure}{fg=osprey_blue} % itemize, enumerate, etc
\setbeamercolor{section in toc}{fg=osprey_blue} % TOC sections

\setbeamerfont{section in toc}{size=\footnotesize}
\setbeamerfont{subsection in toc}{size=\scriptsize}
\setbeamerfont{subsubsection in toc}{size=\scriptsize}

\setbeamertemplate{enumerate item}[default]
\setbeamertemplate{itemize/enumerate body begin}{\small}
\setbeamertemplate{itemize/enumerate subbody begin}{\tiny}
\setbeamertemplate{itemize/enumerate subsubbody begin}{\tiny}

\setbeamerfont{description body}{size=\small}

\setminted{linenos, escapeinside=@@, fontsize=\footnotesize, tabsize=4, xleftmargin=1cm}
\usemintedstyle{emacs}

\tikzset{declare function={f(\x)=0.25*(\x-2.5)^3-0.75*\x+5);}}


\title[Serial vs. Parallel Implementation]{Serial vs. Parallel Implementation of \\
                                           Definite Integrals using the Trapezoid Rule}
\subtitle{\vspace{1em}\footnotesize Presented by:\vspace{-2em}}
\author[\textbf{Group X}]{\textbf{Group X} \\
                          \footnotesize \textit{Jason Gardner, Matthew Thomas, \& William L Thomson Jr}}
\institute[UNF]{University of North Florida \\
                Parallel Computing COP6616 \\
                Instructor Scott Piersall}
\date{\today}

\begin{document}

\begin{frame}[plain]
	\titlepage
\end{frame}

\begin{frame}[allowframebreaks]{Outline}
	\tableofcontents
	\vfill
\end{frame}

\section{Integration}

\begin{frame}{Area Under The Curve}
\begin{multicols}{2}

\begin{tikzpicture}[line join=round,line cap=round]
	\begin{scope}
	\draw[-latex] (-0.5,0) -- (5,0) node[below] {$x$};
	\draw[-latex] (0,-0.5) -- (0,5) node[left]  {$y$};
	\node at (2,3.5) {$f(x)=0.1 x^2+2$};
    \foreach\i in {1, 2, 3, 4}
        \node at (\i, -0.25){\i};
    \node at (1, -0.7) {$a$};
    \node at (4, -0.7) {$b$};

	\end{scope}

    \fill [osprey_green, domain=1:4, variable=\x]
      (1, 0)
      -- plot ({\x}, {0.1*\x*\x+2})
      -- (4, 0)
      -- cycle;
      
\draw [thick,cyan]
      plot [domain=0:5, variable=\x] ({\x}, {0.1*\x*\x+2}) node[right] at (1.5,2)
      {};

      
	\end{tikzpicture}

\vfill\null
\columnbreak

\begin{itemize}
\item In calculus, definite integrals calculate the area "under the curve".
\pause
\item Suppose $F(x)$ is the anti-derivative of $f(x)$. That is,

\[F(x)= \int f(x) dx\]
\pause
Then, the area under $f(x)$, between $a$ and $b$, is given by the definite integral:

\[\int_a^b f(x) dx = F(b) - F(a).\]
\end{itemize}
\end{multicols}
\end{frame}


\begin{frame}{Area Under The Curve Example}
\begin{multicols}{2}

\begin{tikzpicture}[line join=round,line cap=round]
	\begin{scope}
	\draw[-latex] (-0.5,0) -- (5,0) node[below] {$x$};
	\draw[-latex] (0,-0.5) -- (0,5) node[left]  {$y$};
	\node at (2,3.5) {$f(x)=0.1*x^2+2$};
    \foreach\i in {1, 2, 3, 4}
        \node at (\i, -0.25){\i};
    \node at (1, -0.7) {$a$};
    \node at (4, -0.7) {$b$};

	\end{scope}

    \fill [osprey_green, domain=1:4, variable=\x]
      (1, 0)
      -- plot ({\x}, {0.1*\x*\x+2})
      -- (4, 0)
      -- cycle;
      
\draw [thick,cyan]
      plot [domain=0:5, variable=\x] ({\x}, {0.1*\x*\x+2}) node[right] at (1.5,2)
      {};

      
	\end{tikzpicture}

\vfill\null
\columnbreak

\begin{itemize}
\item The anti-derivative for this example is easy to find:

\[F(x)=\int (\frac{1}{10} x^2 +2) dx\]
\pause
\[= \frac{1}{30} x^3 +2x +C \]
\end{itemize}
\end{multicols}
\end{frame}

\begin{frame}{Area Under The Curve Example}
\begin{multicols}{2}

\begin{tikzpicture}[line join=round,line cap=round]
	\begin{scope}
	\draw[-latex] (-0.5,0) -- (5,0) node[below] {$x$};
	\draw[-latex] (0,-0.5) -- (0,5) node[left]  {$y$};
	\node at (2,3.5) {$f(x)=0.1*x^2+2$};
    \foreach\i in {1, 2, 3, 4}
        \node at (\i, -0.25){\i};
    \node at (1, -0.7) {$a$};
    \node at (4, -0.7) {$b$};

	\end{scope}

    \fill [osprey_green, domain=1:4, variable=\x]
      (1, 0)
      -- plot ({\x}, {0.1*\x*\x+2})
      -- (4, 0)
      -- cycle;
      
\draw [thick,cyan]
      plot [domain=0:5, variable=\x] ({\x}, {0.1*\x*\x+2}) node[right] at (1.5,2)
      {};

      
	\end{tikzpicture}

\vfill\null
\columnbreak

\begin{itemize}
\item So the area under the curve, over the interval from 1 to 4, is:
\[\int_1^4 (\frac{1}{10} x^2 +2) dx = F(4)-F(1)\]
\pause
\[=\frac{1}{30}4^3+2(4)-\frac{1}{30}1^3-2(1)\]

\[=8.1\]
\pause
Exactly!
\vfill\null
\end{itemize}
\end{multicols}
\end{frame}

\begin{frame}{The Need For A Numerica Method}

\begin{multicols}{2}
%The following picture is from
%https://tikz.net/dynamics_pendulum_block/
%which has the Tikz code that could be adapted
\begin{figure}
    \centering
    \includegraphics[width=0.7\linewidth]{pendulum.png}
    %\caption{Enter Caption}
    %\label{fig:enter-label}
\end{figure}

\vfill\null
\columnbreak

A pendulum is released with an initial angle $\theta_0$. As it swings, the period, $T$, is given by the following:
\pause
\[T(\theta_0)=2\sqrt{2}\sqrt{\frac{L}{g}}\int_0^{\theta_0} \frac{d\theta}{\sqrt{cos \theta - cos \theta_0}}\]

where $L$ is the length of the pendulum and $g$ is the gravitational constant.
\pause
\\
\\
\textbf{There is no closed-form anti-derivative for this integral!}
\end{multicols}
\end{frame}

\begin{frame}{Numerical Methods for Integration}
\begin{itemize}
\item Gaussian Quadrature
\item Romberg Algorithm
\item Simpson's Rule
\item \textbf{Trapezoid Rule}
\end{itemize}
\end{frame}



\section{Trapezoid Rule}
\begin{frame}{Trapezoid Rule: Formula}
\[ \int_a^b f(x)dx \approx \frac{h}{2} (f(x_0) + f(x_n)) + \sum_{i=1}^{n-1} f(x_i) \]
where $h = (b-a)/n, x_i = a + ih$

Given p processes, each process can work on n/p intervals

Note: for simplicity will assume $n/p$ is an integer

\begingroup
\renewcommand{\arraystretch}{1.5}
\begin{center}
\begin{tabular}{ l l }
 \hline
 process & interval \\ 
 \hline
 0 & $[a, a + \frac{n}{p}h]$ \\ 
 1 & $[a + \frac{n}{p}h, a + 2\frac{n}{p}h]$ \\ 
 ... & ... \\  
 $p-1$ & $[a + (p - 1)\frac{n}{p}h, b]$ \\
 \hline
\end{tabular}
\end{center}
\endgroup

\end{frame}

\begin{frame}{Trapezoid Rule: Integration}
	\begin{multicols}{2}
	Area $ = \frac{1}{2}[f(a) + f(b)](b-a)$ \\
	
	\begin{spacing}{1.25}
		Area $ = \frac{1}{2}[f(x_0) + f(x_1)](x_1-x_0) +$ \\
		\phantom{x}\hspace{2.5em} $ \frac{1}{2}[f(x_1) + f(x_2)](x_2-x_1) +$
		\phantom{x}\hspace{2.5em} $ \frac{1}{2}[f(x_2) + f(x_3)](x_3-x_2) +$ \\

		\hspace{2em} $ = \frac{h}{2}[f(x_0) + f(x_1) +$ \\
		\phantom{x}\hspace{3.5em} $f(x_1) + f(x_2) + $ \\
		\phantom{x}\hspace{3.5em} $f(x_2) + f(x_3)]$
	\end{spacing}

	\begin{tikzpicture}[line join=round,line cap=round]
		\begin{scope}
		\draw[-latex] (-0.5,0) -- (5,0) node[below] {$x$};
		\draw[-latex] (0,-0.5) -- (0,5) node[left]  {$y$};
		\node at (3.25,3.5) {$y=f(x)$};
		\foreach\i in {0,1,2,3}
		{
			\pgfmathsetmacro\j{1*\i+1}
			\coordinate (x\i) at (\j,0);
			\coordinate (y\i) at (\j,{f(\j)});
			\node[below]        at (x\i) {$x_\i$};
		}
		\node[yshift=-0.75cm] at (x0) {\strut$a$};
		\node[yshift=-0.75cm] at (x3) {\strut$b$};
		\end{scope}
		\foreach\i in {0,1,2,3}
		{
			\pgfmathtruncatemacro\j{\i+1}
			\ifnum\i<3
				\filldraw[osprey_green] (x\i) -- (y\i) -- (y\j) -- (x\j);
				\draw[gray] (x\i) -- (y\i) -- (y\j) -- (x\j);
			\fi
			\fill (y\i) circle (1pt);
		}
		\draw[thick,cyan] plot[domain=0.5:4.5,samples=41] (\x,{f(\x)});
	\end{tikzpicture}
	\end{multicols}
	\vspace{-2em}
	\hspace{2em} $ = \frac{h}{2}[f(x_0) + 2f(x_1) + 2f(x_2) + f(x_3)]$
\end{frame}

\begin{frame}{Trapezoid Rule: Integration Precision}
	\begin{multicols}{2}
	Precision can be increased by increasing the number of partitions. \\
	
	Error in approximation decreases as the partition step size decreases. \\
	
	Partition area calculations are independent.
	\begin{tikzpicture}[line join=round,line cap=round]
		\begin{scope}
		\draw[-latex] (-0.5,0) -- (5,0) node[below] {$x$};
		\draw[-latex] (0,-0.5) -- (0,5) node[left]  {$y$};
		\node at (3.25,3.5) {$y=f(x)$};
		\foreach\i in {0,1,2,3,...,8}
		{
			\pgfmathsetmacro\j{0.5*\i+0.5}
			\coordinate (x\i) at (\j,0);
			\coordinate (y\i) at (\j,{f(\j)});
		}
		\foreach\i in {0,1,2,3}
			\node[below]          at (x\i) {$x_\i$};
		\node[below]          at (x7) {$x_{n-1}$};
		\node[below]          at (x8) {$x_n$};
		\node[yshift=-0.75cm] at (x0) {\strut$a$};
		\node[yshift=-0.75cm] at (x8) {\strut$b$};
		\end{scope}
		\foreach\i in {0,1,2,3,...,8}
		{
			\pgfmathtruncatemacro\j{\i+1}
			\ifnum\i<8
				\filldraw[osprey_green] (x\i) -- (y\i) -- (y\j) -- (x\j);
				\draw[gray] (x\i) -- (y\i) -- (y\j) -- (x\j);
			\fi
			\fill (y\i) circle (1pt);
		}
		\draw[thick,cyan] plot[domain=0.5:4.5,samples=41] (\x,{f(\x)});
	\end{tikzpicture}
	\end{multicols}
\end{frame}

\begin{frame}{Trapezoid Rule: Accuracy}
\begin{itemize}
\item Numerical methods of integration, such as using the Trapezoid Rule, only provide an estimate of the actual area.
\item More sub-intervals may lead to greater accuracy.
\item How many sub-intervals are needed to achieve a desired accuracy? There's a way to find ahead of time!
\item Unfortunately, we need to find the second derivative of the function being integrated, and we must do some analysis on that.
\item Suppose $E_T$ is the difference between our estimate and the actual value of the integral. This is the Total Error.
\item The following gives an upper bound for $|E_T|$
\[|E_T|\leq(b-a)\frac{h^2}{12}M\]
\end{itemize}
\end{frame}

\begin{frame}{Trapezoid Rule: Accuracy}
\begin{itemize}
\item Suppose $E_T$ is the difference between our estimate and the exact value of the integral (which we don't have to know). This is the Total Error.
\item The following gives an upper bound for $|E_T|$ when integrating $f(x)$ on the interval $[a,b]$:
\[|E_T|\leq(b-a)\frac{h^2}{12}M\]
where $h$ is the sub-interval width, and $M$ is the maximum value
\item We specify a maximum bound for $|E_T|$, so after we determine $M$, we can solve for $h$ and subsequently solve for $n$, the number of sub-intervals we need for our desired accuracy.
\end{itemize}
\end{frame}

\begin{frame}{Example: Finding Number of Sub-intervals Needed}
\end{frame}



\begin{frame}{Trapezoid Rule: Integration Step Size}
\end{frame}

\begin{frame}{Trapezoid Rule: Integration Error}
\animategraphics[autoplay,loop,width=\linewidth - 0.5cm]{1}{assets/Trapezium2-}{0}{19}
\end{frame}

\begin{frame}{Trapezoid Rule: Parallelizing}
	\begin{enumerate}
		\item Partition problem solution into tasks.
		\item Identify communication channels between tasks. (MPI)
		\item Aggregate tasks into composite tasks.
		\item Map composite tasks to cores.
	\end{enumerate}

	\begin{tikzpicture}[line join=round,line cap=round]
		\begin{scope}
		\draw[-latex] (-0.5,0) -- (10,0) node[below] {$x$};
		\draw[-latex] (0,-0.5) -- (0,4) node[left]  {$y$};
		\node at (6.5,3.5) {$y=f(x)$};
		\foreach\i in {0,1,2,3,4}
		{
			\pgfmathsetmacro\j{0.9*\i+0.9}
			\coordinate (x\i) at (\j * 2,0);
			\coordinate (y\i) at (\j * 2,{f(\j)});
			\node[below]       at (x\i) {$x_\i$};
		}
		\foreach\i in {0,1,2,3,4}
		{
			\pgfmathtruncatemacro\j{\i+1}
			\ifnum\i<4
				\filldraw[osprey_green] (x\i) -- (y\i) -- (y\j) -- (x\j);
				\draw[gray] (x\i) -- (y\i) -- (y\j) -- (x\j);
				\node at (\j * 1.75 + 1,2) {Task \i};
				\node at (\j * 1.75 + 1,1.5) {Core \i};
			\fi
			\fill (y\i) circle (1pt);
		}
		\draw[thick,cyan] plot[domain=0.5:4.5,samples=41,tension=1] (\x * 2,{f(\x)});
		\end{scope}
	\end{tikzpicture}
\end{frame}

\section{Serial Implementation}
\begin{frame}{Serial Implementation: Approach}
\end{frame}

\begin{frame}[containsverbatim]{Serial Implementation: Code}
\begin{minted}{c}
/* Function being integrated using @$f(x) = x^2$@*/
float f(float x) {
	return x*x;
}
float trapezoid_rule(float a, float b, int n, float h) {
	float integral; /* Store result in integral */
	float x;
	int i;
	integral = (f(a) + f(b))/2.0;
	x = a;
	for ( i = 1; i <= n-1; i++ ) {
		x = x + h;
		integral = integral + f(x);
	}
	return integral*h;
}
\end{minted}
\end{frame}

\section{Parallel Implementation}
\begin{frame}{Parallel Implementation: Approach}
	\begin{itemize}
		\item Split the interval $[a,b]$ up among the $p$ processes.
		\item Each process will estimate the integral $f(x)$ over its subinterval.
		\item To estimate the total integral, the processes’ local calculations are added.
		\item Suppose that the $n$ trapezoids are divided evenly across $p$ processes.
		\item Process $q$ will estimate the integral over the interval \\
		$[a + q\frac{nh}{p}, a + (q + 1)\frac{nh}{p}]$
		
		\item Each process needs the following information:
		\begin{itemize}
			\item Number of processes, $p$.
			\item Its rank.
			\item The entire interval of integration, $[a, b]$.
			\item The number of subintervals, $n$.
		\end{itemize}
	\end{itemize}
\end{frame}

\begin{frame}{Parallel Implementation: Code}
\end{frame}

\begin{frame}{References}
\begin{itemize}
    \item Pendulum graphic: Licensed under a Creative Commons Attribution-ShareAlike 4.0 International License with Copyright 2021 – TikZ.net
    \item Pendulum formula: Kharab, A., \& Guenther, R. B. (2018). An introduction to numerical methods: A MATLAB Approach. Chapman \& Hall/CRC Numerical Analysis and Scientific Computing Series.
\end{itemize}
\end{frame}

\end{document}
