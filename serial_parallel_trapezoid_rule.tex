\documentclass[t]{beamer}
\usetheme{Boadilla}

\usepackage[english]{babel}
\usepackage[font=tiny]{caption}
\usepackage[outputdir=/home/wlt/.cache/gummi/]{minted}
\usepackage{multicol}
\usepackage{multirow}
\usepackage{setspace}
\usepackage{tikz}

\newcommand{\eg}{\textit{e}.\textit{g}., }
\newcommand{\ie}{\textit{i}.\textit{e}., }
\newcommand{\heading}[1]{{\it{\usebeamercolor[fg]{frametitle} #1}}}

\definecolor{osprey_blue}{cmyk}{1, .76, .1, .65}
\definecolor{osprey_gray}{cmyk}{.36, .29, .28, 0}
\definecolor{osprey_green}{cmyk}{.59, .02, 1, 0}

\setbeamercolor*{palette primary}{bg=osprey_blue, fg = white}
\setbeamercolor*{palette secondary}{bg=osprey_gray, fg = white}
\setbeamercolor*{palette tertiary}{bg=osprey_blue, fg = white}
\setbeamercolor{structure}{fg=osprey_blue} % itemize, enumerate, etc
\setbeamercolor{section in toc}{fg=osprey_blue} % TOC sections

\setbeamerfont{section in toc}{size=\footnotesize}
\setbeamerfont{subsection in toc}{size=\scriptsize}
\setbeamerfont{subsubsection in toc}{size=\scriptsize}

\setbeamertemplate{enumerate item}[default]
\setbeamertemplate{itemize/enumerate body begin}{\small}
\setbeamertemplate{itemize/enumerate subbody begin}{\tiny}
\setbeamertemplate{itemize/enumerate subsubbody begin}{\tiny}

\setbeamerfont{description body}{size=\small}

\tikzset{declare function={f(\x)=0.25*(\x-2.5)^3-0.75*\x+5);}}


\title[Serial vs. Parallel Implementation]{Serial vs. Parallel Implementation of \\
                                           Definite Integrals using the Trapezoid Rule}
\subtitle{\vspace{1em}\footnotesize Presented by:\vspace{-2em}}
\author[\textbf{Group X}]{\textbf{Group X} \\
                          \footnotesize \textit{Jason Gardner, Matthew Thomas, \& William L Thomson Jr}}
\institute[UNF]{University of North Florida \\
                Parallel Computing COP6616 \\
                Instructor Scott Piersall}
\date{\today}

\begin{document}

\begin{frame}[plain]
	\titlepage
\end{frame}

\begin{frame}[allowframebreaks]{Outline}
	\tableofcontents
	\vfill
\end{frame}

\section{Trapezoid Rule}
\begin{frame}{Trapezoid Rule: Formula}
\[ \int_a^b f(x)dx \approx \frac{h}{2} (f(x_0) + f(x_n)) + \sum_{i=1}^{n-1} f(x_i) \]
where $h = (b-a)/n, x_i = a + ih$

Given p processes, each process can work on n/p intervals

Note: for simplicity will assume $n/p$ is an integer

\begingroup
\renewcommand{\arraystretch}{1.5}
\begin{center}
\begin{tabular}{ l l }
 \hline
 process & interval \\ 
 \hline
 0 & $[a, a + \frac{n}{p}h]$ \\ 
 1 & $[a + \frac{n}{p}h, a + 2\frac{n}{p}h]$ \\ 
 ... & ... \\  
 $p-1$ & $[a + (p - 1)\frac{n}{p}h, b]$ \\
 \hline
\end{tabular}
\end{center}
\endgroup

\end{frame}

\begin{frame}{Trapezoid Rule: Integration}
	\begin{multicols}{2}
	Area $ = \frac{1}{2}[f(a) + f(b)](b-a)$ \\
	
	\begin{spacing}{1.25}
		Area $ = \frac{1}{2}[f(x_0) + f(x_1)](x_1-x0) +$ \\ 
		\phantom{x}\hspace{2.5em} $ \frac{1}{2}[f(x_1) + f(x_2)](x_2-x1) +$
		\phantom{x}\hspace{2.5em} $ \frac{1}{2}[f(x_2) + f(x_3)](x_3-x2) +$ \\

		\hspace{2em} $ = \frac{h}{2}[f(x_0) + f(x_1) +$ \\
		\phantom{x}\hspace{3.5em} $f(x_1) + f(x_2) + $ \\
		\phantom{x}\hspace{3.5em} $f(x_2) + f(x_3)]$
	\end{spacing}

	\begin{tikzpicture}[line join=round,line cap=round]
		\begin{scope}
		\draw[-latex] (-0.5,0) -- (5,0) node[below] {$x$};
		\draw[-latex] (0,-0.5) -- (0,5) node[left]  {$y$};
		\node at (3.25,3.5) {$y=f(x)$};
		\foreach\i in {0,1,2,3}
		{
			\pgfmathsetmacro\j{1*\i+1}
			\coordinate (x\i) at (\j,0);
			\coordinate (y\i) at (\j,{f(\j)});
		}
		\foreach\i in {0,1,2,3}
		\node[below]        at (x\i) {$x_\i$};
		\node[yshift=-0.75cm] at (x0) {\strut$a$};
		\node[yshift=-0.75cm] at (x3) {\strut$b$};
		\end{scope}
		\foreach\i in {0,1,2,3}
		{
			\pgfmathtruncatemacro\j{\i+1}
			\ifnum\i<3
			\filldraw[osprey_green] (x\i) -- (y\i) -- (y\j) -- (x\j);
			\draw[gray] (x\i) -- (y\i) -- (y\j) -- (x\j);
			\fi
			\fill (y\i) circle (1pt);
		}
		\draw[thick,cyan] plot[domain=0.5:4.5,samples=41] (\x,{f(\x)});
	\end{tikzpicture}
	\end{multicols}
	\vspace{-2em}
	\hspace{2em} $ = \frac{h}{2}[f(x_0) + 2f(x_1) + 2f(x_2) + f(x_3)]$
\end{frame}

\end{document}